\begin{name}
{Sở GD Nghệ An}
{ĐỀ THI THỬ TỐT NGHIỆP THPT--NĂM 2022 }
\end{name}
\Opensolutionfile{ans}[nghean]
\begin{ex}
 Họ tất cả các nguyên hàm của hàm số $y=\cos 3 x$ là
\choice
{$\sin 3 x+C$}
{$\sin 3 x+C$}
{$-\dfrac{\sin 3 x}{3}+C$}
{$\dfrac{\sin 3 x}{3}+C$}
\end{ex}

\begin{ex}
 Cho hình trụ có bán kính đáy $r=5 \mathrm{~cm}$ và khoảng cách giữa hai đáy bằng $8 \mathrm{~cm}$. Diện tích xung quanh của hinh tru là
\choice
{$120 \pi \mathrm{cm}^{2}$}
{$160 \pi \mathrm{cm}^{2}$}
{$40 \pi \mathrm{cm}^{2}$}
{$80 \pi \mathrm{cm}^{2}$}
\end{ex}

\begin{ex}
 Đường tiệm cận đứng của đồ thị hàm số $y=\dfrac{x-2}{x-1}$ có phương trình là
\choice
{$x=2$}
{$y=1$}
{$y=2$}
{$x=1$}
\end{ex}

\begin{ex}
 Mệnh đề nào sau đây đúng ?
\choice
{$C_{20}^{5}=\dfrac{20 \text {!}}{5 !}$}
{$C_{20}^{5}=\dfrac{20 !}{15 !}$}
{$C_{20}^{5}=\dfrac{20 !}{5 ! \cdot 15 !}$}
{$C_{20}^{5}=\dfrac{20 !}{5.15}$}
\end{ex}

\begin{ex}
 Cho hình chóp $S. A B C D$ có đáy là hình vuông $A B C D$ cạnh bằng 3, cạnh bên $S A$ vuông góc với mặt phẳng đáy. Biết $S A=3$, tính thể tích $V$ của khối chóp $S \cdot A B C D$.
\choice
{$V=9$}
{$V=\dfrac{1}{3}$}
{$V=27$}
{$V=3$}
\end{ex}

\begin{ex}
 Hàm số nào sau đây có bảng biến thiên như hình dưới ?

\choice
{$y=x^{3}-3 x$}
{$y=-x^{3}+3 x^{2}+1$}
{$y=x^{3}-3 x^{2}-1$}
{$y=-x^{3}+3 x$}
\end{ex}

\begin{ex}
 Bất phương trình $\log _{2}(3 x-1)>3$ có nghiệm là
\choice
{$\dfrac{1}{3}<x<3$}
{$x>\dfrac{10}{3}$}
{$x>3$}
{$x<3$}
\end{ex}

\begin{ex}
 Cho số phức $z$ thỏa mãn: $z(2-i)+13 i=1$. Số phức liên hợp của $z$ là
\choice
{$\bar{z}=3-5 i$}
{$\bar{z}=3+5 i$}
{$\bar{z}=-3+5 i$}
{$\bar{z}=-3-5 i$}
\end{ex}

\begin{ex}
 Số giao điềm của đồ thị hàm số $y=x^{4}-5 x^{2}+4$ với trục hoành là
\choice
{3}
{2}
{4}
{1}
\end{ex}

\begin{ex}
 Cho tích phân $I=\int_{1}^{e} \dfrac{3 \ln x+1}{x} \mathrm{~d} x$. Nếu đặt $t=\ln x$ thì
\choice
{$I=\int_{1}^{e} \dfrac{3 t+1}{t} \mathrm{~d} t$}
{$I=\int_{0}^{1} \dfrac{3 t+1}{\mathrm{e}^{t}} \mathrm{~d} t$}
{$I=\int_{0}^{1}(3 t+1) \mathrm{d} t$}
{$I=\int_{1}^{e}(3 t+1) \mathrm{d} t$}
\end{ex}

\begin{ex}
 Điểm $M$ trong hình vẽ là điểm biểu diễn của số phức $z$.

Phần ảo của số phức $z$ là
\choice
{$-4 i$}
{$-3$}
{$-4$}
{3}
\end{ex}

\begin{ex}
 Trong không gian $O x y z$, cho đường thẳng $d: \dfrac{x-6}{1}=\dfrac{y-2}{-2}=\dfrac{z+4}{-3}$, vectơ nào dưới đây là vectơ chi phương của đường thẳng $d$ ?
\choice
{$\overrightarrow{u_{4}}=(1; 2;-3)$}
{$\overrightarrow{u_{2}}=(-6;-2; 4)$}
{$\overrightarrow{u_{1}}=(6; 2;-4)$}
{$\overrightarrow{u_{3}}=(1;-2;-3)$}
\end{ex}

\begin{ex}
 Cho hàm số $f(x)$ có đạo hàm liên tục trên đoạn $[1; 3]$ thỏa mãn $f(1)=2$ và $f(3)=9$. Tính $I=\int_{1}^{3} f^{\prime}(x) \mathrm{d} x.$
\choice
{$I=7$}
{$I=2$}
{$I=18$}
{$I=11$}
\end{ex}

\begin{ex}
 Cho $a$ là số thực dương. Mệnh đề nào dưới đây đúng ?
\choice
{$\log _{2} a^{3}=3 \log _{2} a$}
{$\log _{2} a^{3}=\dfrac{1}{3} \log _{2} a$}
{$\log _{2} a^{3}=\dfrac{3}{2} \log a$}
{$\log _{2} a^{3}=3 \log a$}
\end{ex}

\begin{ex}
 Phần thực của số phức $z=3+4 i$ là
\choice
{3}
{4}
{$-3$}
{$4 i$}
\end{ex}

\begin{ex}
 Dãy số nào sau đây là cấp số cộng ?
\choice
{$-1; 1;-1; 1$}
{$3; 5; 7; 10$}
{$4; 6; 8; 10$}
{$4; 8; 16; 32$}
\end{ex}

\begin{ex}
 Thể tích của khối cầu bán kính $R$ bằng
\choice
{$\dfrac{4}{3} \pi R^{3}$}
{$4 \pi R^{2}$}
{$\dfrac{1}{3} \pi R^{3}$}
{$\dfrac{3}{4} \pi R^{3}$}
\end{ex}

\begin{ex}
 Cho số phức $z_{1}=1+3 i$ và $z_{2}=3-4 i$. Môđun của số phức $w=z_{1}-z_{2}$ là
\choice
{$|w|=\sqrt{17}$}
{$|w|=53$}
{$|w|=17$}
{$|w|=\sqrt{53}$}
\end{ex}

\begin{ex}
 Cho hàm số $y=f(x)$ xác định, liên tục trên $\mathbb{R}$ và có đồ thị như hình vẽ

Hàm số có bao nhiêu điểm cực tiểu ?
\choice
{1}
{3}
{0}
{2}
\end{ex}

\begin{ex}
 Hàm số $y=x^{3}-3 x^{2}+5$ đồng biến trên khoảng nào dưới đây ?
\choice
{$(0; 2)$}
{$(-\infty; 0)$}
{$(-\infty; 2)$}
{$(0;+\infty)$}
\end{ex}

\begin{ex}
 Họ tất cả các nguyên hàm của hàm số $f(x)=x^{4}+x^{2}$ là
\choice
{$4 x^{3}+2 x+C$}
{$x^{5}+x^{3}+C$}
{$\dfrac{1}{5} x^{5}+\dfrac{1}{3} x^{3}+C$}
{$x^{4}+x^{2}+C$}
\end{ex}

\begin{ex}
 Đường cong như hình vẽ là đồ thị của hàm số nào

\choice
{$y=-x^{4}-2 x^{2}+1$}
{$y=-x^{4}+1$}
{$y=x^{4}-2 x^{2}-1$}
{$y=-x^{4}+2 x^{2}+1$}
\end{ex}

\begin{ex}
 Với a là một số thực dương tùy ý, $a^{\dfrac{2}{3}}$ bằng
\choice
{$\sqrt[3]{a}$}
{$\sqrt{a^{3}}$}
{$\sqrt[6]{a}$}
{$\sqrt[3]{a^{2}}$}
\end{ex}

\begin{ex}
 Đạo hàm của hàm số $y=7^{x}$ là
\choice
{$y^{\prime}=7^{x} \ln 7$}
{$y^{\prime}=\dfrac{7^{x}}{\ln 7}$}
{$y^{\prime}=7^{x}$}
{$y^{\prime}=x.7^{x-1}$}
\end{ex}

\begin{ex}
 Cho hàm số $y=f(x)$ có đạo hàm $f^{\prime}(x)=(x-2)^{2}, \forall x \in \mathbb{R}$. Giá trị lớn nhất của hàm số đã cho trên đoạn $[0; 5]$ bằng
\choice
{$f(2)$}
{$f(4)$}
{$f(0)$}
{$f(5)$}
\end{ex}

\begin{ex}
 Số nghiệm thực của phương trình $2^{x^{2}-x-4}=\dfrac{1}{4}$ là
\choice
{0}
{2}
{1}
{3}
\end{ex}

\begin{ex}
 Trong không gian $O x y z$, cho hai điểm $A(1; 1;-1), B(2; 3; 2)$. Vectơ $\overrightarrow{A B}$ có tọa độ là
\choice
{$(3; 5; 1)$}
{$(1; 2; 3)$}
{$(3; 4; 1)$}
{$(-1;-2; 3)$}
\end{ex}

\begin{ex}
 Cho $\int_{0}^{1} f(x) d x=2$ và $\int_{1}^{6} f(x) d x=5$, khi đó $\int_{0}^{6} f(x) d x$ bằng
\choice
{7}
{10}
{$3$}
{6}
\end{ex}

\begin{ex}
 Trong không gian $O x y z$, cho mặt cầu $(S)$ có tâm $I(-1; 2; 1)$, bán kính $R=2$. Phương trình của mặt cầu $(S)$ là
\choice
{$(S):(x+1)^{2}+(y-2)^{2}+(z-1)^{2}=4$}
{$(S):(x+1)^{2}+(y-2)^{2}+(z-1)^{2}=2$}
{$(S):(x-1)^{2}+(y+2)^{2}+(z+1)^{2}=2$}
{$(S):(x-1)^{2}+(y+2)^{2}+(z+1)^{2}=4$}
\end{ex}

\begin{ex}
 Trong không gian $O x y z$, mặt phẳng đi qua điểm $M(1; 2; 3)$ và song song với mặt phẳng $(P): x-2 y+z-3=0$ có phương trình là
\choice
{$x-2 y+z=0$}
{$x-2 y+z+3=0$}
{$x+2 y+3 z=0$}
{$x-2 y+z-8=0$}
\end{ex}

\begin{ex}
 Trong không gian $O x y z$, cho đường thẳng $\Delta$ đi qua điểm $M(1; 2; 2)$ song song với mặt phẳng $(P): x-y+z+3=0$ đồng thời cắt đường thẳng $d: \dfrac{x-1}{1}=\dfrac{y-2}{1}=\dfrac{z-3}{1}$. Hỏi đường thẳng $\Delta$ đi qua điểm nào sau đây ?
\choice
{$L(1;-3; 7)$}
{$K(4; 5; 2)$}
{$F(2; 3; 4)$}
{$E(2; 3;-2)$}
\end{ex}

\begin{ex}
 Cho hình chóp $S. A B C$ có đáy $A B C$ là tam giác vuông cạnh huyền $B C=a$. Hình chiếu vuông góc của $S$ lên $(A BC)$ trùng với trung điểm $B C$. Biết $S B=a$, tính số đo góc giữa $S A$ và $(A BC)$
\choice
{$60^{\circ}$}
{$30^{\circ}$}
{$90^{\circ}$}
{$45^{\circ}$}
\end{ex}

\begin{ex}
 Cho hình hình hộp chữ nhật $A B C D \cdot A_{1} B_{1} C_{1} D_{1}$ có đáy $A B C D$ là hình vuông cạnh bằng 2. Góc giữa $A C_{1}$ và $B B_{1}$ bằng $30^{\circ}$. Tính thể tích khối hộp chữ nhật $A B C D \cdot A_{1} B_{1} C_{1} D_{1}$.
\choice
{$8 \sqrt{6}$}
{$\dfrac{8 \sqrt{6}}{3}$}
{$\dfrac{\sqrt{6}}{12}$}
{$4 \sqrt{6}$}
\end{ex}

\begin{ex}
 Cho hàm số $y=f(x)=\left\{\begin{array}{ll}x^{2}+1 & \text {khi} x<0 \\ e^{x}. & \text {khi} x \geq 0\end{array}.\right.$ Biết $\int_{-2}^{2} f(x) d x=a \cdot e^{2}+b \cdot e+c(a, b, c \in \mathbb{Q}).$ Tính tổng $a+b+c$.
\choice
{$\dfrac{17}{3}$}
{$\dfrac{8}{3}$}
{3}
{$\dfrac{14}{3}$}
\end{ex}

\begin{ex}
 Trong không gian $O x y z$, cho điểm $I(2; 3; 4)$. Mặt cầu $(S)$ có tâm $I$ cắt trụcc $O x$ tại hai điểm phân biệt $A, B$ sao cho diện tích của tam giác $L A B$ bằng 60. Phương trình của mặt cầu $(S)$ là
\choice
{$(x-2)^{2}+(y-3)^{2}+(z-4)^{2}=169$}
{$(x-2)^{2}+(y-3)^{2}+(z-4)^{2}=225$}
{$(x-2)^{2}+(y-3)^{2}+(z-4)^{2}=144$}
{$(x-2)^{2}+(y-3)^{2}+(z-4)^{2}=196$}
\end{ex}

\begin{ex}
 Cho hai số phức $z_{1}, z_{2}$ thoả mãn $z_{1}+z_{2}-2=z_{1} \cdot z_{2}-2=0$. Hãy tính $\left|z_{1}\right|^{2}+\left|z_{2}\right|^{2}$.
\choice
{2}
{4}
{8}
{1}
\end{ex}

\begin{ex}
Để kiểm tra sản phẩm của một công ty sữa, người ta gửi đến bộ phận kiểm nghiệm $ 5  $ hộp sữa cam, $ 4$ hộp sữa nho và $3$ hộp sữa dâu. Bộ phận kiểm nghiệm chọn ngẫu nhiên $3$ hộp sữa để phân tích mẫu. Xác suất để $3$ hộp sữa được chọn đủ cả $ 3$ loại là
\choice
{$\dfrac{3}{11}$}
{$\dfrac{1}{5}$}
{$\dfrac{3}{7}$}
{$\dfrac{1}{6}$}
\end{ex}

\begin{ex}
 Số nghiệm của phương trình $\log _{2} x \cdot \log _{3}(2 x-1)=2 \log _{2} x$ là
\choice
{1}
{3}
{2}
{0}
\end{ex}

\begin{ex}
 Bất phương trình $3^{x^{2}} \leq 15^{x}$ có bao nhiêu nghiệm nguyên dương
\choice
{1}
{2}
{4}
{3}
\end{ex}

\begin{ex}
 Trong không gian $O x y z$, cho điểm $A(-4;-2; 4)$ và đường thẳng $d: \left\{\begin{array}{l}x=-3+2 t \\ y=1-t \\ z=-1+4 t\end{array}\right.$. Viết phương trình mặt phẳng $(P)$ đi qua $A$ và vuông góc với đường thằng $d$.
\choice
{$(P): 2 x-y+4 z-10=0$}
{$(P):-2 x+y-4 z-10=0$}
{$(P):-3 x+y-z+10=0$}
{$(P): 2 x+y-4 z+10=0$}
\end{ex}

\begin{ex}
 Cho số phức $z$ thỏa mãn điều kiện $|z+2-6 i|=|\bar{z}-3+5 i|$ và số phức $z_{1}$ có phần thực bằng phần ảo. Giá trị nhỏ nhất của biểu thức $\left|z-z_{1}+z_{1}^{2}\right|$ là
\choice
{$\dfrac{3 \sqrt{26}}{26}$}
{$\dfrac{9}{8}$}
{$\dfrac{1}{5}$}
{$\dfrac{\sqrt{26}}{26}$}
\end{ex}

\begin{ex}
 Cho hàm số $y=f(x)$ có đạo hàm trên $\mathbb{R}$. Đồ thị hàm số $y=f^{\prime}(-x)$ được cho bởi hình vẽ sau:
Điều kiện của tham số $m$ để bất phương trình $f\left(\sqrt{2-x^{2}}\right) \leq m$ nghiệm đúng với mọi $x \in[-\sqrt{2}; \sqrt{2}]$ là
\choice
{$m \geq f(0)$}
{$m<f(\sqrt{2})$}
{$m \geq f(\sqrt{2})$}
{$m>f(\sqrt{2})$}
\end{ex}

\begin{ex}
 Cho đồ thị của hàm số $y=f(x)=a x^{4}-b^{2} x^{2}+c(a, b, c \in \mathbb{R})$ là đường cong ở hình vẽ:

Số các giá trị nguyên của $m$ để phương trình $x f(\sqrt{x})=(2 m+2) x^{2}-\left(m^{2}-5\right) x-1$ có hai nghiệm $x_{1}, x_{2}$ thỏa mãn $x_{1}<1<x_{2}$ ?
\choice
{2}
{$1.$}
{5}
{4}
\end{ex}

\begin{ex}
 Cho hình lăng trụ $A BC.A^{\prime} B^{\prime} C^{\prime}$ có đáy $A B C$ là tam giác đều cạnh bằng $1, B B^{\prime}$ tạo với đáy một góc $60^{\circ}$, hình chiếu của $A^{\prime}$ lên mặt phẳng $(A B C$ trùng với trung điểm $H$ của cạnh $B C$. Tính khoảng cách từ $C$ đến mặt phẳng $\left(A B B^{\prime}\right)$.
\choice
{$\dfrac{2 \sqrt{13}}{13}$}
{$\dfrac{3 \sqrt{13}}{13}$}
{$\dfrac{\sqrt{13}}{13}$}
{$\dfrac{4 \sqrt{13}}{13}$}
\end{ex}

\begin{ex}
 Cho $a, b, c>1$ thỏa mãn $6 \log _{2 a b} c \geq 1+\log _{2 b} c$. $\log _{a} c$ và biết phương trình $c^{x^{2}+1}=a^{x}$ có nghiệm. Giá trị lớn nhất của biểu thức $P=\log _{a}\left(2 b c^{2}\right)$ bằng $\dfrac{\dot{m}+\sqrt{n}}{p}$ trong đó $m, n, p$ là các số nguyên dương và $\dfrac{m}{p}$ là phân số tối giản. Giá trị của $m+n+p$ bằng.
\choice
{48}
{60}
{64}
{56}
\end{ex}

\begin{ex}
 Cho hình chóp $S. A B C D$ có đáy $A B C D$ là hình chữ nhật, $A B=a, A D=2 a$ và $S A$ vuông góc với đáy. Gọi $M$ là trung điểm của cạnh $S C$, biết khoảng cách từ $M$ đến mặt phẳng $(SBD)$ bằng $\dfrac{a}{4}$. Tính thể tích khối chóp  $S.A B M$.
\choice
{$\dfrac{a^{3} \sqrt{11}}{66}$}
{$\dfrac{a^{3} \sqrt{11}}{33}$}
{$\dfrac{2 a^{3} \sqrt{11}}{33}$}
{$\dfrac{4 a^{3} \sqrt{11}}{33}$}
\end{ex}

\begin{ex}
 Cho hàm số $f(x)=x^{3}+a x^{2}+b x(a, b \in \mathbb{R})$. Biết hàm số $g(x)=f(x)-\dfrac{2}{3} f^{\prime}(x)+\dfrac{1}{6} f^{\prime \prime}(x)$ có hai điểm cực trị là $x=1, x=\dfrac{1}{3}$. Với mỗi $t$ là hằng số tùy ý thuộc đoạn $[0; 1]$ gọi $S_{1}$ là diện tích hình phẳng giới hạn bời các đường: $x=0, y=f(t), y=f(x)$ và $S_{2}$ là diện tích hình phẳng giới hạn bởi các đường: $y=f(x), y=f(t), x=1$. Biểu thức $P=8 S_{1}+4 S_{2}$ có thể nhận được bao nhiêu giá trị là số nguyên ?
\choice
{6}
{4}
{2}
{8}
\end{ex}

\begin{ex}
 Óng thép mạ kẽm (độ dày của ống thép là hiệu số bán kính mặt ngoài và bán kính mặt bên trong của ông thép). Nhà máy quy định giá bán cho các loại ống thép dưa trên cân nặng của các ống thép đó. Biết rằng thép ống có giá là 24700 đồng/kg và khối lượng riêng của thép là $7850 \mathrm{~kg} / \mathrm{m}^{3}$. Một đại lí thép mua về 1000 ống thép loại có đường kính ngoài là $60 \mathrm{~mm}$, độ dày là $3 \mathrm{~mm}$ và có chiều dài là $6 \mathrm{~m}$. Hãy tính số tiền mà đại lí bỏ ra để mua 1000 ống thép nói trên (làm tròn đến ngàn đồng).
\choice
{623789000 đồng}
{624977000 đồng}
{624980000 đồng}
{623867000 đồng}
\end{ex}

\begin{ex}
 Trong không gian $O x y z$ cho các điểm $A(4; 0; 0), B(0; 8; 0), C(0; 0; 12), D(-1; 7;-9)$ và $M$ là một điểm nằm ngoài mặt cầu $(S)$ ngoại tiếp tứ diện $O A B C$. Các đường thẳng $M A, M B, M C$, $M O$ lần lượt cắt mặt cầu $(S)$ tại các điềm $A^{\prime}, B^{\prime}, C^{\prime}, O^{\prime}$(khác  A, B, C, O) sao cho $\dfrac{M A}{M A^{\prime}}+\dfrac{M B}{M B^{\prime}}+\dfrac{M C}{M C^{\prime}}+\dfrac{M O}{M O^{\prime}}=4$. Tìm giả trị nhỏ nhất của $M D+M O$.
\choice
{$8 \sqrt{3}$}
{$10 \sqrt{3}$}
{$9 \sqrt{3}$}
{$11 \sqrt{3}$}
\end{ex}

\begin{ex}
 Cho hàm số $f(x)=x^{3}-3 x^{2}+3 x-1$. Biết hàm số $g(x)=a x^{4}+b x^{2}+c(a, b, c \in \mathbb{R}, a \neq 0)$ nhận $x=1$ là điểm cực trị. Số điểm cực trị của hàm số $y=g(f(x))$ là
\choice
{5}
{6}
{$4.$}
{3}
\end{ex}
\Closesolutionfile{ans}